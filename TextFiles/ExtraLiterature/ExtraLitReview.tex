\documentclass[12pt]{article} % 

\usepackage{amssymb}
\usepackage{amsthm}
\usepackage{amsfonts}
\usepackage{amsmath}
% \usepackage{cite}



\setlength{\oddsidemargin}{-0.15in}
\setlength{\topmargin}{-0.5in}
\setlength{\textwidth}{6.5in}
\setlength{\textheight}{9in}

\newcommand{\N}{\mathbb{N}}
\newcommand{\Z}{\mathbb{Z}}
\newcommand{\R}{\mathbb{R}}

\renewcommand{\v}{\mathbf{v}}

\renewcommand\le{\leqslant}
\renewcommand\ge{\geqslant}



\begin{document} 

%\noindent
\begin{center}
\textbf{Literature Review}
\end{center}
\bigskip

%\begin{itemize}
%\item
%\end{itemize}

\section*{A Survey on Weapon Target Allocation Models \\ \cite{Ghanbari2021ASO}}
\begin{itemize}
\item Two key components of command and control are: weapon target allocation (WTA) and threat evaluation.
\item Resource allocation is stochastic/uncertain with regard to the WTA problem.
\item The WTA component of the WTA problem can be considered in 3 parts: response planning, response execution, outcome assessment.
\item There exists three basic models:
\begin{center}
\begin{enumerate}
    \item[Basic Model 1:] For maximizing damage to enemy (minimize expected target values $F$), we have \[\text{min}(\ F\ )\ =\ \sum_{i=1}^{|T|}V_i\ \Pi_{k=1}^{|W|}(1 - P_{ik})^{x_ik}\]
    This is the general WTA formula.
    \item[Basic Model 2:] For allocation of available units to maximize expected total protection value $J$, we have \[\text{max}(\ J\ )\ =\ \sum_{j=1}^{|A|}\omega_j\ \Pi_{i \in G_j}(1 - \pi_{ij}\Pi_{k=1}^{|W|}(1 - P_{ik})^{x_ik}\]
    \item[Basic Model 3:] This is the model for Dynamic WTA at stage $t$ given the total expected combat value of surviving assets \[\text{max}(\ J_tX^t\ )\ =\ \sum_{j=1}^{|A(t)|}\omega_j\ \Pi_{i=1}^{|T(t)|}\left[1 - \pi_{ij}\Pi_{h=t}^S\Pi_{k=1}^{|W(t)|}(1-p_{ik}(h))^{x_{ih}(h)}\right]\]
    \item[]
    \begin{center}
    \begin{tabular}{|p{4.25cm}  p{8cm}|}
    		\hline
    		\textbf{Variable Definitions} & \ \\
            \hline
            Sets & \\
    		\hline
    		$T_i$ & Set of detected threats $i = 1, 2, \cdots, I$. \\
            \ & \ \\
    		$w_k$ & Set of resources $k = 1, 2, \cdots, K$. \\
            \ & \ \\
    		$A_j$ & Set of assets $j = 1, 2, \cdots, J$. \\
            \ & \ \\
    		$S$ & Set of engagement stages, $s = 1, 2, \cdots, S$. \\
            \ & \ \\
            $A(t), T(t), W(t)$ & Set of current "defended assets, hostile targets, and available weapons during stage $t$, respectively."\\
            \hline
            Parameters & \ \\
            \hline
    		$P_{ik}$ & Estimated effectiveness/probability that weapon $w_k \in W$ neutralizes threat $T_i \in T$ if assigned to it.\\
            \ & \ \\
            $\pi_{ij}$ & Estimated probability threat $T_i \in T$ destroys asset $A_j \in A$. \\
            \ & \ \\
            $V_{ik}$ & Threat value of the threat-asset pair $(T_i, A_j)$. \\
            \ & \ \\
            $\omega_j$ & Protection value of asset $A_j$. \\
            \ & \ \\
            $C_{ik}$ & Resource usage cost for assigning $w_k$ to $T_i$. \\
            \hline
            Variables & \ \\
            \hline
            $X_{ik}$ & Is $1$ if resource $w_k$ is assigned to $T_i$, $0$ otherwise. \\
            \ & \ \\
            $\left[X_{ik}^s\right]_{I\times K}$ & Decision matrix at stage $s$. \\
            \ & \ \\
            $h$ & Index of stages $t, \cdots, S$. \\
    		\hline
    \end{tabular}
    \end{center}
\end{enumerate}
\end{center}

\item Dynamic WTA (DWTA) suffer from curse of dimensionality.
\item WTA problem has two perspectives: $\textit{single platform perspective} \text{ and }\\\textit{force coordination perspective.}$ The former is single platform defending one asset against incoming threats, the latter is a command and control platform defending multiple assets.
\item Within these perspectives exist two paradigms: $\textit{threat-by-threat} \text{ and }\\\textit{multi-threat.}$ The former being sequential targeting and the latter being parallel targeting.
\item There also exists two different prioritizations of defense, as shown by basic models 1 and 2.
\item Static WTA (SWTA) constraints: $X_{ik} \in \{0, 1\} \forall i\in\{1, 2, \cdots, |T|\},\forall k\in\{1, 2, \cdots, |W|\}$ given the equations: $\sum_{i = 1}^{|T|}X_{ik} = 1 \ \ \ \forall k\in\{1, 2, \cdots, |W|\}$ if each firing unit must be assigned a target and $\sum_{i = 1}^{|T|}X_{ik} \le 1 \ \ \ \forall k\in\{1, 2, \cdots, |W|\}$ otherwise, with $X_{ik}$ being a target $i$ being assigned to a resource $k$ in the constraint matrix $X$. \\
\item Dynamic WTA (DWTA) problems have more constraints as follows:
\vspace{-0.5cm}
\begin{center}
\begin{enumerate}
    \begin{center}
    \item[Weapon multi-target constraint: ] This constraint describes multi-target systems. As each multi-target system can also be considered as separate systems, $n_k = 1 \forall k\in\{1, 2, \cdots, W\}$. \[\sum_{i = 1}^{|T|}x_{ik}(t) \le n_k \ \ \ \forall t\in\{1, 2, \cdots, S\}, \forall k\in\{1, 2, \cdots, |W|\}\]
    \item[Strategy constraint: ] This constraint limits system-usage cost per target at stage $t$. $m_i$ depends on performance of available resource $k$ on target $i$. For missile systems, $m_i = 1$, and for artillery systems, $m_1 \ge 1$. \[\sum_{k = 1}^{|W|}x_{ik}(t) \le m_i \ \ \ \forall t\in\{1, 2, \cdots, S\}, \forall i\in\{1, 2, \cdots, |T|\}\]
    \item[Resource constraint: ] This constraint governs over ammunition availability. \[\sum_{t=1}^S\sum_{i=1}^{|T|}x_{ik}(t)\le N_k, \ \ \ \forall k\in\{1, 2, \cdots, |W|\}\]
    \item[Engagement feasibility constraint: ] This constraint is over the resource-target relationship: if a target $i$ can be hit by a resource $k$ at stage $t$, then $f_{ik}(t) = 1$, and $f_{ik}(t) = 0$ otherwise.
    \begin{align*}
        x_{ik}(t) \le f_{ik}(t), && &\forall t\in\{1, 2, \cdots, S\}, \forall i \in \{1, 2, \cdots, |T|\}
    \end{align*}
    \end{center}
\end{enumerate}
\end{center}
\end{itemize}


\section*{Optimization of decision support system based on \\ three-stage threat evaluation and resource management \cite{Naseem2017OptimizationOD}}
\begin{itemize}
    \item For the interception of theater ballistic missiles (TBMs), the U.S. Air Force noted that there exists two options for the WTAP: the Markov decision processes, an extension of the Markov chain, and approximate dynamic programming (ADP) for WTAPs involving TBMs.
    \item For the latter option, there exist two different algorithmic approaches: approximate value iteration and approximate policy iteration (API); the paper uses the latter.
    \item The API algorithmic strategy maps the system state -- includes incoming target amount, current asset health, and interceptor health -- to reaction fire against incoming targets, specifically how many interceptors to assign to each incoming target.
    \item MDP Formula:\\
    Let $\Gamma = \{1, 2, \cdots, T\},\ T \le \infty$, where the number of decision epochs $T$ is random and follows a geometric distribution with parameter $0 \le \gamma < 1$. \\Asset status component $a_t = (a_{ti})_{i\in A}\equiv (a_{t1}, a_{t2}, \cdots, a_{t|A|}$, where set of all assets $A = \{1, 2, \cdots, |A|\}$ and health of asset $a_{ti} \in \{0, 0.25, 0.5, 0.75, 1\}$, with $0$ being destroyed and $1$ being undamaged. \\
    Let resource inventory component $R_t = (R_{ti})_{i\in A} \equiv (R_{t1}, R_{t2}, \cdots, R_{t|A|})$, with status element $R_{ti} \in \{0,1,\cdots r_i\}$, with
    \item The MDP formula is too complex to be summarized, please refer to pages 7-10 of the paper.
    \item There are two value function approximations for API: least squares policy evaluation (LSPE) and least squares temporal difference (LSTD).
    \item \begin{center}
        \begin{enumerate}
        \end{enumerate}
    \end{center}
\end{itemize}



\section*{An approximate dynamic programming approach for comparing firing policies in a networked air defense environment \cite{Summers2020AnAD}}
Lorem ipsum 3 $\ldots$


\section*{Threat Evaluation In Air Defense Systems Using Analytic Network Process \cite{Unver2019ThreatEI}}
Lorem ipsum 4 $\ldots$

%\section*{The SSA-BP-based potential threat prediction for aerial target considering commander emotion \\ \cite{WANG20222097}}
%Lorem ipsum 5 $\ldots$

\newpage
\begin{center}
\bibliographystyle{apalike}
\bibliography{ExtraLiterature}
\end{center}

\end{document}
