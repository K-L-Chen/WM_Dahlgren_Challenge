\documentclass[12pt]{article} % 

\usepackage{amssymb}
\usepackage{amsthm}
\usepackage{amsfonts}
\usepackage{amsmath}
% \usepackage{cite}



\setlength{\oddsidemargin}{-0.15in}
\setlength{\topmargin}{-0.5in}
\setlength{\textwidth}{6.5in}
\setlength{\textheight}{9in}

\newcommand{\N}{\mathbb{N}}
\newcommand{\Z}{\mathbb{Z}}
\newcommand{\R}{\mathbb{R}}

\renewcommand{\v}{\mathbf{v}}

\renewcommand\le{\leqslant}
\renewcommand\ge{\geqslant}

\begin{document} 
%\noindent
\begin{center}
\textbf{Literature Review}
\end{center}
\bigskip

%\begin{itemize}
%\item
%\end{itemize}

\section*{A Survey on Weapon Target Allocation Models \\ \cite{Ghanbari2021ASO}}
\begin{itemize}
\item Two key components of command and control are: weapon target allocation (WTA) and threat evaluation.
\item Resource allocation is stochastic/uncertain with regard to the WTA problem.
\item The WTA component of the WTA problem can be considered in 3 parts: response planning, response execution, outcome assessment.
\item There exists three basic models: 
\begin{enumerate}
    \item[Basic Model 1] For maximizing damage to enemy (minimize expected target values $F$), we have \[\text{min}(\ F\ )\ =\ \sum_{i=1}^{|T|}V_i\ \Pi_{k=1}^{|W|}(1 - P_{ik})^{x_ik}\]
    This is the general WTA formula.
    \item[Basic Model 2] For allocation of available units to maximize expected total protection value $J$, we have \[\text{max}(\ J\ ) = \sum_{j-1}^{|A|}\omega_j\ \Pi_{i \in G_j}(1 - \pi_{ij}\Pi_{k=1}^{|W|}(1 - P_{ik})^{x_ik}\]
    \item[Basic Model 3]
\end{enumerate}
\end{itemize}


\section*{Optimization of decision support system based on \\ three-stage threat evaluation and resource management \cite{Naseem2017OptimizationOD}}
Lorem ipsum 2 $\ldots$



\section*{An approximate dynamic programming approach for comparing firing policies in a networked air defense environment \cite{Summers2020AnAD}}
Lorem ipsum 3 $\ldots$


\section*{Threat Evaluation In Air Defense Systems Using Analytic Network Process \cite{Unver2019ThreatEI}}
Lorem ipsum 4 $\ldots$

\section*{The SSA-BP-based potential threat prediction for aerial target considering commander emotion \\ \cite{WANG20222097}}
Lorem ipsum 5 $\ldots$

\newpage
\begin{center}
\bibliographystyle{apalike}
\bibliography{ExtraLiterature}
\end{center}

\end{document}
