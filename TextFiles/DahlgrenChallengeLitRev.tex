\documentclass[12pt]{article} % 

\usepackage{amssymb}
\usepackage{amsthm}
\usepackage{amsfonts}
\usepackage{amsmath}
% \usepackage{cite}



\setlength{\oddsidemargin}{-0.15in}
\setlength{\topmargin}{-0.5in}
\setlength{\textwidth}{6.5in}
\setlength{\textheight}{9in}

\newcommand{\N}{\mathbb{N}}
\newcommand{\Z}{\mathbb{Z}}
\newcommand{\R}{\mathbb{R}}

\renewcommand{\v}{\mathbf{v}}

\renewcommand\le{\leqslant}
\renewcommand\ge{\geqslant}

\begin{document} 
%\noindent
\begin{center}
\textbf{Literature Review}
\end{center}
\bigskip

\section*{The Weapon-Target Assignment Problem \\ \cite{wtap_kline_2019}}

\begin{itemize}
    \item As $\uparrow$ quantity and quality of missiles, effective allocation research emerged.
    \item Weapon Target Assignment (WTA) aka Missile Allocation Problem (MAP) $\Leftrightarrow$ minimize probabiilty of a missile destroying a protected assignment
    \item Sometimes offense perspective OR defense perspective 
    \item WTA $\rightarrow$ Static WTA (SWTA) or Dynamic WTA (DWTA)
    \item SWTA:
    \\ input: num. of incoming missiles (targets), num. of interceptors (weapons), probabilities of destroying targets
    \\ output: how many of each weapon type to shoot at each target

    \item DWTA includes time as a dimension. Two variants: two-stage and shoot-look-shoot

    \begin{itemize}
        \item Two-stage DWTA: 
        \\ stages/input: 1. SWTA and 2. probability distribution of various kinds of targets 
        \\ output: 1. allocation of weapons and 2. how many weapons to reserve to minimize prob. of destruction
        
        \item Shoot-kill-Shoot DWTA:
        \\ replicates SWTA too, but enables observation of leakers: target that maybe survived the initial engagement for a subsequent engagement
        \\ solution: allocation of weapons and reservation of weapons to rengage any leakers
    \end{itemize}

    \item WTA is NP-Complete, so majority of solutions seek near-optimal solutions in real-time, or fast-enough solutions before the adversary reaches their goals.
    \item These solutions use heuristics or have exact solutions applied to variants of the WTA problem

\end{itemize}

\subsection*{Formulations}

Notation:
\begin{itemize}
    \item $p_{i j}$ : the probability weaponi destroys target $j$
    \item $q_{i j} (= 1 - p_{i j})$ : the probability weapon $i$ fails to destroys target $j$
    \item $V_j$ : the destructive value of target $j$
    \item $x_{i j}$ : the number of weapons of type $i$ assigned to target $j$
    \item $K$ : the number of protected assets
    \item $a_k$ : the value of asset $k$
    \item $n$ : the number of targets
    \item $m$ : the number of weapon types
    \item $w_i$ : the number of weapons of type $i$
    \item $c_{i j}$ : a cost parameter for assigning a weapon of type $i$ to target $j$
    \item $\mathcal{F}$ : the set of feasible assignments
    \item $\gamma_{j k}$ : the probability target $j$ destroys asset $k$
    \item $s_j$ : the maximum number of weapons that can be assigned to target $j$
    \item $t$ : the number of stages
\end{itemize}


{\bf SWTA's main formulation}:
$$
\begin{array}{ll}
\min & \sum_{j=1}^n V_j \prod_{i=1}^m q_{i j}^{x_{i j}} \\
\text { s.t. } & \sum_{j=1}^n x_{i j} \leq w_i \text {, for } i=1, \ldots, m \\
& x_{i j} \in \mathbb{Z}_{+}, \text {for } i=1, \ldots, m, j=1, \ldots, n
\end{array}
$$

In English: find the best assignment of number of weapons, across all types,
that minimizes the survival rate of the targets, with higher emphasis of those with more destructive value.
The ``such that (s.t.)'' requriements indicate that we must respect our supply capacity of weapons and 
that we must allocate at least weapon of each type to all targets.\\

Other formulations:
They make simplifying assumptions, such as assuming that the probability of using any weapon
to destroy a target is the same or that there is only a capacity of one weapon of each type. Maybe there is 
one weapon type per target. All of these different assumptions simplify optimization in some way, but obviously
each has its pros and cons in terms of optimizability and applicability to real-world scenarios.  

{\bf DWTA's main formulation}:
- I'm not going to really elaborate on the math; it gets substantially more complicated. I think an important 
takeaway is that because SWTA was already intractable with its large amount of permutations to begin with, 
DWTA is certainly intractable as well, given how it's basically just a stack of SWTAs. 
- Feel free to look at the math yourself, but be willing to spend a lot of attention and time. I just did not think it was worth it.
- Antoher important takeaway: it's a must to use approximation methods. 


\subsection*{Exact Algorithms for SWTA}

{\bf Maximum Marginal Return (MMR)}

This algorithm assums that the probabiilty of kill for any weapon target to target $j$ is the same.

Then, the optimal solution is:

\begin{enumerate}
\item Assign $x_{ij} = 1$ where $\{i, j\} \in argmax(V_j p_{ij})$
\item $V_j \leftarrow V_j(1-p_{ij})$
\item $p(i, \cdot) \leftarrow 0$ and $p(\cdot, j) \leftarrow 0$
\item Repeat until all weapons have been assigned
\end{enumerate}

Note that it's best to divide the weapons evenly across all targets when there is only one 
probability of kill, regardless of weapon or target.\\

\noindent {\bf On Solving the Original SWTA Formulation}

When you try exhaustive searches: a problem with $9$ weapons and $8$ targets
take $13$ min to run to completion, and adding one additional target takes $43.7$ min. to run to completion. 
Thus, there is a combinatorial explosion in run time as a function of the problem size.

\noindent {\bf Brief Mention of Other Algorithms}
\begin{itemize}
    \item Branch-and-bound
    \item Lower bounding strategies (generalized network flow, MMR, and minimum cost flow)
    \item Linear integer programming
    \item Joint Munition Effectiveness Manual (JMEM)
\end{itemize}

\subsection*{Exact Algorithms for DWTA}
\begin{itemize}
    \item Mathematics-based (Burr et al. (1985) \& Soland (1987) \& Hosein (1989))
    \item Concave Adaptive Value Estimation (CAVE) with modified MMR (aka MMR Plus Algorithm)
    \item Dynamic Programming
\end{itemize}

\subsection*{Heruistics for SWTA}
All the above solutions focus on the optimal solution, but heuristic algorithms focus on real-time solutions.

\begin{itemize}
    \item Genetic algorithms 
    \item Very Large Scale Neighborhood (VLSN) search metahueristic
    \item Ant Colony Optimization
    \item Integer relaxed NLP and rounding schemes 
    \item Neural networks
    \item Network-flow-based construction heruistic 
    \item Simulated Annealing (SA)
    \item Variable Neighbor Search (VNS)
    \item Tabu Search 
    \item Particle Swarm Optimization (PSO)
    \item Lagrangian relaxation Branch and Bound
    \item Hungarian Algorithm
\end{itemize}

\subsection*{Heruistics for DWTA}

\begin{itemize}
    \item ALIAS algorithm 
    \item Decomposition algorithm 
    \item Virtual permutations
    \item Rule-based
    \item Hungarian Algorithm
    \item Neuro-dynamic programming to obtain near optimal policies. Optimal policies are obtained through dynamic programming
\end{itemize}

\subsection*{Discussion}
The branch and bound algorithm and the genetic algorithm are both widely used in the literature. Bogdanowicz (2012)'s algorithm or 
Xin et al. (2010)'s rule-based hueristic efficiently exploit the special structure of the problem.


\section*{Applying reinforcement learning to the weapon assignment problem in air defence \cite{rl_wa_airDefence_mouton_2011}}

\subsection*{Abstract}
\begin{itemize}
    \item Monte Carlo control algorithm with Exploring Starts (MCES)
    \item Q-learning: an off-policy temporal-difference (TD) learning-control algorithm
    \item These algos. used in SIMPLIFIED version of weapon assignment (WA) problem
\end{itemize}


\subsection*{Introduction}
\begin{itemize}
    \item RL uses framework to define interaction between learning agent and environment in terms of states, actions, rewards
    \item RL agent must discover by trial and error to get highest reward 
    \item No exact method exists for the WA problem, evne when it is realtively small-sized 
    \item Compare to threat evaluation (TE), WA decisions are more quantifiable
    \item This article discusses whether RL is suitable for WA. 
    
    Flow: RL \& related work overview, greater context $\rightarrow$ command and control (C2), threat evaluation and weapon assignment (TEWA)
    $\rightarrow$ how WA problem was modeled and how MCES \& Q-learning were applied $\rightarrow$ compare experimental results 
    $\rightarrow$ ideas for future study
\end{itemize}

\subsection*{Overview of reinforcement learning and related work}
\begin{itemize}
    \item Azak \& Bayrak implemented agents for TWEA problems of C2 systems for decision-making.
    \item They sought to optimize decision-making performance for multi-armed-platform-TEWA problems 
    \item RL can be seen as a Markov Decision Process (MDP). This allows agent designers to work with systems 
    in their current state, without worrying about how it came to be in that state in the first place. 
    \item RL loop: agent given info. about env. through sensory input $\rightarrow$ action $\rightarrow$ reward
    \item RL algos. maps each world state to agent actions
\end{itemize}

\subsection*{Command and control}
\begin{itemize}
\item C2 system = facilities, equipment, communications, procedures, personnel essential for planning, directing, and controlling 
for a commander.
\item It involves observation, orientation, decision, and action. 
\item Observation: gather info. relevant to decision
\item Orientation: cognitive effort; breaking down problem into sub-problems and matching each with emergency plan $\rightarrow$ overall action plan 
\item Decision: whether to execute the action plan 
\item Action: execution of chosen course of action or plan, e.g. physical attack/movement, order issuance, sensor maintenance for better future observations
\item I don't think the rest of this subsection and the next one on threat evaluation and weapon assignment is that important for the lit. review.
\end{itemize}

\subsection*{Modeling the weapon assignment problem}
\begin{itemize}
    \item Asset to defend is at the center of the grid.
    \item 4 missile stations defended the asset. Once these stations were placed, they remained fixed.
    \item Threat takes straight line towards asset, and flies in from one of the cells. Once a threat reaches the asset, it eliminates the asset.
    \item Threat stays in specific cell while all the weapons get a turn to shoot at the threat. If all weapons miss, the threat moves one cell closer.
    \item If threat on weapon position, no shots can be fired. Thus, the threat gets a free pass to move one cell closer.
    \item 
\end{itemize}


\section*{Optimization of Weapon-Target Pairings Based on Kill Probabilities \cite{killProbs_bogdanowicz_2013}}
Lorem ipsum 3 $\ldots$


\section*{A New Approach to Weapon-Target Assignment in Cooperative Air Combat \cite{swarmHarmony_chang_2017}}
Lorem ipsum 4 $\ldots$


\section*{A Coordinated Air Defense Learning System Based on Immunized Classifier Systems \cite{immunized_nantogma_2021}}
Lorem ipsum 5 $\ldots$


\section*{The state-of-the-art review on resource allocation problem using artificial intelligence methods on various computing paradigms \cite{resourceAlloc_joloudari_2022}}
Lorem ipsum 6 $\ldots$


\bibliographystyle{apalike}
\bibliography{LitRev}


asdfasdfafdasfd

\end{document}
